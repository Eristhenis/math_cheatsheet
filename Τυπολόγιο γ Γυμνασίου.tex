\documentclass[a4paper,landscape,10pt]{article}

\usepackage{eristhenis_math}
\usepackage{multicol}


\begin{document}
\gr


%-----------------------------------------------------------
%-----------------------------------------------------------
%\newpage
%-----------------------------------------------------------
%-----------------------------------------------------------
\begin{center}
    \Large \textbf{ΤΥΠΟΛΟΓΙΟ γ' τάξης Γυμνασίου} \\
    \rule{.6\textwidth}{0.5pt}
\end{center}


\begin{multicols}{3}
%-----------------------------------------------------------
\myTitle{ΔΥΝΑΜΕΙΣ}

\begin{minipage}{.3\linewidth}
\mybullet $a^{\ \mu} \cdot a^v=a^{\ \mu+v}$ \\
\mybullet $(a^{\ \mu})^v=a^{\ \mu\cdot v}$ \\
\mybullet $\dfrac{1}{a^{\ \mu}}=a^{-\mu}$ \\
\end{minipage}
\hfill\vline\hfill
\begin{minipage}{.3\linewidth}
\mybullet $(a\cdot\beta)^v=a^v\cdot \beta^v$ \\
\mybullet $\left(\dfrac{a}{\beta}\right)^v=\dfrac{a^v}{\beta^v}$ \\
\mybullet $\dfrac{a^{\ \mu}}{a^v}=a^{\ \mu-v}$ 
\end{minipage}


%-----------------------------------------------------------
\myTitle{ΡΙΖΕΣ}

\mybullet Κάτω από τη ρίζα βάζω ΜΟΝΟ θετικό αριθμό \\
\begin{minipage}{.45\linewidth}
\mybullet $\sqrt{a}\ge0$ \\
\mybullet $\sqrt{a^2}=|a|$ \\
\mybullet $\left(\sqrt{a}\right)^2=a$ \\
\mybullet $\sqrt{a}=0\Leftrightarrow a=0$ \\
\mybullet $\sqrt{a+\beta}\ne\sqrt{a}+\sqrt{\beta}$
\end{minipage}\hfill\vline\hfill
\begin{minipage}{.45\linewidth}
\mybullet $\sqrt{a}\ne0\Leftrightarrow a\ne0$ \\
\mybullet $\sqrt{a\cdot\beta}=\sqrt{a}\cdot \sqrt{\beta}$ \\
\mybullet $\sqrt{\dfrac{a}{\beta}}=\dfrac{\sqrt{a}}{\sqrt{\beta}}$ 
\end{minipage} 


%-----------------------------------------------------------
\myTitle{Μονώνυμα}
Για το μονώνυμο $2x^3y^4$ \\
\mybullet To $2$ λέγεται συντελεστής \\
\mybullet To $x^3y^4$ λέγεται κύριο μέρος \\
\mybullet To $3$ λέγεται βαθμός ως προς $x$ \\
\mybullet To $4$ λέγεται βαθμός ως προς $y$ \\
\mybullet To $7$ λέγεται βαθμός ως προς όλες τις μεταβλητές


%-----------------------------------------------------------
\myTitle{Πολυώνυμα με μία μεταβλητή}
\mybullet Βαθμός του πολυωνύμου λέγεται ο μεγαλύτερος βαθμός 
των μονωνύμων που το αποτελούν \\
\mybullet Βαθμός σταθερού πολυωνύμου είναι το μηδέν \\
\mybullet Βαθμός του μηδενικού πολυωνύμου δεν ορίζεται \\


%-----------------------------------------------------------
\myTitle{ΤΑΥΤΟΤΗΤΕΣ}
\mybullet $(a\pm\beta)^2=a^2\pm2a\beta+\beta^2$ \\
\mybullet $a^2-\beta^2=(a+\beta)(a-\beta)$ \\
\mybullet $(a\pm\beta)^3=a^3\pm3a^2\beta+3a\beta^2\pm\beta^3$ %\\
%\mybullet $(a-\beta)^3=a^3-3a^2\beta+3a\beta^2-\beta^3$ 

%-----------------------------------------------------------
\myTitle{$ax=\beta$} 
\mybullet αν $a=0$ και $\beta=0$, τότε είναι ταυτότητα \\
\mybullet αν $a=0$ και $\beta\ne0$, τότε είναι αδύνατη \\
\mybullet αν $a\ne0$, τότε έχει μοναδική λύση $x=\dfrac{\beta}{a}$ 

%-----------------------------------------------------------
\myTitle{ $x^2=\beta$}
\mybullet Αν $\beta<0$, είναι αδύνατη \\
\mybullet Αν $\beta=0$, τότε $x=0$ \\
\mybullet Αν $\beta>0$, τότε $x=\sqrt{\beta}$ ή $x=-\sqrt{\beta}$ 

%-----------------------------------------------------------
\myTitle{Διάταξη}
\mybullet $a<\beta\Rightarrow a+\gamma<\beta+\gamma$ \\
\mybullet $a<\beta\Rightarrow a-\gamma<\beta-\gamma$ \\
\mybullet Αν $\gamma>0$, τότε $a<\beta\Rightarrow a\gamma<\beta\gamma$ \\
\mybullet Αν $\gamma<0$, τότε $a<\beta\Rightarrow a\gamma>\beta\gamma$ \\

%-----------------------------------------------------------
\myTitle{Ανισώσεις α' βαθμού}
\mybullet Απαλοιφή παρονομαστών \\
\mybullet Βγάζω τις παρενθέσεις \\
\mybullet Χωρίζω γνωστούς από αγνώστους \\
\mybullet Αναγωγή ομοίων όρων \\
\mybullet Διαιρώ με τον συντελεστή του αγνώστου (αν είναι αρνητικός, 
αλλάζει η φορά της ανίσωσης) 


%-----------------------------------------------------------
\myTitle{Τριώνυμο $ax^2+\beta x+\gamma$ με $a\ne0$}
\mybullet Υπολογίζω τη $\Delta=\beta^2-4a\cdot\gamma$ \\
\mybullet Αν $\Delta<0$, δεν παραγοντοποιείται \\
\mybullet Αν $\Delta=0$, τότε $ax^2+\beta x+\gamma=a\left(x-\dfrac{-\beta}{2a}\right)^2$ \\
\mybullet Αν $\Delta>0$, τότε $ax^2+\beta x+\gamma=a(x-x_1)(x-x_2)$, όπου 
$x_1=\dfrac{-\beta+\sqrt{\Delta}}{2a}$ και $x_2=\dfrac{-\beta-\sqrt{\Delta}}{2a}$ 


%-----------------------------------------------------------
\myTitle{Εξίσωση $ax^2+\beta x+\gamma=0$ με $a\ne0$}

%\mybullet $ax^2+\beta x+\gamma=0$ με $a\ne0$\\
\mybullet Υπολογίζω τη $\Delta=\beta^2-4a\cdot\gamma$ \\
\mybullet $\Delta<0\Rightarrow$ αδύνατη \\
\mybullet $\Delta=0\Rightarrow$ μία ρίζα διπλή. $x_{1,2}=\dfrac{-\beta}{2a}$ \\
\mybullet $\Delta>0\Rightarrow$ δύο άνισες ρίζες. $x_{1,2}=\dfrac{-\beta\pm\sqrt{\Delta}}{2a}$ 

%-----------------------------------------------------------
\myTitle{$ax+\beta y=\gamma$ με $a\ne0$ ή $\beta\ne0$}

\mybullet παριστάνει πάντα ευθεία. \\
\mybullet αν $a=0$ η ευθεία είναι οριζόντια. \\
\mybullet αν $\beta=0$ η ευθεία είναι κατακόρυφη. \\
\mybullet αν $a\ne0$ και $\beta\ne0$ η ευθεία έχει κλίση. \\
\mybullet Για τη γραφική παράσταση αρκούν δύο σημεία τα οποία ενώνω. \\
\mybullet Για να βρω σημείο της ευθείας, βάζω τιμή στο $x$ (ή στο $y$) 
και βρίσκω το αντίστοιχο $y$ (ή το $x$). \\

%-----------------------------------------------------------
\myTitle{ΣΥΣΤΗΜΑΤΑ ΕΞΙΣΩΣΕΩΝ}

\mybullet Γραφική επίλυση \\
\mybullet Μέθοδος αντικατάστασης \\
\mybullet Μέθοδος αντίθετων συντελεστών 


%-----------------------------------------------------------
\myTitle{ΚΡΙΤΗΡΙΑ ΙΣΟΤΗΤΑΣ ΤΡΙΓΩΝΩΝ}

\mybullet Π--Π--Π \hspace{5mm} Π--Γ--Π \hspace{5mm} Γ--Π--Γ \\
\mybullet Σε ίσα τρίγωνα, απέναντι από ίσες πλευρές βρίσκονται ίσες γωνίες, και αντίστροφα 

%-----------------------------------------------------------
\myTitle{ΘΕΩΡΗΜΑ ΘΑΛΗ}

\mybullet Αν τρεις ή περισσότερες ευθείες τέμνονται από δύο άλλες, 
τότε στις δύο άλλες ορίζονται τμήματα ανάλογα \\

%-----------------------------------------------------------
\myTitle{ΟΜΟΙΑ ΤΡΙΓΩΝΑ}

\mybullet Αν δύο γωνίες ενός τριγώνου είναι αντίστοιχα ίσες με δύο γωνίες του άλλου, τότε τα τρίγωνα είναι όμοια \\
\mybullet Σε όμοια τρίγωνα, απέναντι από ίσες γωνίες βρίσκονται ανάλογες πλευρές \\
\mybullet Σε όμοια τρίγωνα, απέναντι από ανάλογες πλευρές βρίσκονται ίσες γωνίες 



%-----------------------------------------------------------
\myTitle{Τριγωνομετρία}

\noindent
Για το σημείο $A(x,y)$ έχω $\rho=OA=\sqrt{x^2+y^2}$ \\
\mybullet $\hm a=\dfrac{\text{απέναντι κάθετη}}{\text{υποτείνουσα}}=\dfrac{y}{\rho}$ \\
\mybullet $\syn a=\dfrac{\text{προσκείμενη κάθετη}}{\text{υποτείνουσα}}=\dfrac{x}{\rho}$ \\
\mybullet $\ef a=\dfrac{\text{απέναντι κάθετη}}{\text{προσκείμενη κάθετη}}=\dfrac{y}{x}$ \\
\rule{\linewidth}{0.5pt} \\
\begin{minipage}{.45\linewidth}
\mybullet $\hm(180-a)=\hm a$ \\
\mybullet $\syn(180-a)=-\syn a$ \\
\mybullet $\ef(180-a)=-\ef a$
\end{minipage}\hfill\vline\hfill
\begin{minipage}{.45\linewidth}
\mybullet $\hm^2a+\syn^2a=1$ \\
\mybullet $\ef a=\dfrac{\hm a}{\syn a}$
\end{minipage}


\end{multicols}


%\QuestionNumber Να υπολογίσετε τα πεδία ορισμού των συναρτήσεων: $f(x)=\frac{\sqrt{1-|\ln{x}|}}{|x|+x}$, $\ g(x)=\ln{\big(x+\sqrt{x^2+1}\big)}$, $h(x)=\ln{\big(x^2-4|x|-5\big)}$,  $\ s(x)=\sqrt{\ln{(e^x+2)}-x}+\ln{|x|}$, $\ t(x)=\frac{1}{2\syn x-1}$, $\ u(x)=\sqrt{2\hm x-1}$.

\end{document}
