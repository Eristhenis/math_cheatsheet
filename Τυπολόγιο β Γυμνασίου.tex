\documentclass[a4paper,landscape,2pt]{book}

\usepackage{eristhenis_math}
\usepackage{multicol}
\usepackage{qrcode}


\begin{document}
\gr

%-----------------------------------------------------------
%-----------------------------------------------------------
\begin{minipage}{2.5cm}
    \ %\qrcode{https://www.1gps.gr}
\end{minipage}
\begin{minipage}{20cm}
\begin{center}
%    {\LARGE \textbf{1$^\circ$ ΓΥΜΝΑΣΙΟ ΝΕΟΥ ΨΥΧΙΚΟΥ}} \\
%    Λ. Τζαβέλλα 64, Ν. Ψυχικό  -- T.K. 15451 \\
%    τηλ. 210-6715918, {\en site: www.1gps.gr} %{\en e-mail:\ mail@1gym-psych.att.sch.gr} 
{\LARGE \textbf{1$^\circ$ ΓΥΜΝΑΣΙΟ ΧΟΛΑΡΓΟΥ}} \\
    Αετιδέων 48, Χολαργός  -- T.K. 15561 \\
    τηλ. 210-6537178, {\en site: gym1cholarg.blogspot.com} %{\en e-mail:\ mail@1gym-psych.att.sch.gr} 

\bigskip
%\begin{center}
    {\Large \textbf{Ωρολόγιο Πρόγραμμα Διδασκαλίας}}
%\end{center}
\end{center}
%\rule{\textwidth}{0.5pt}
\end{minipage}
\begin{minipage}{2.5cm}
    \qrcode{https://gym1cholarg.blogspot.com} %www.1gps.gr}
\end{minipage}


\bigskip
\begin{tabular}{|c|p{4cm}|p{4cm}|p{4cm}|p{4cm}|p{4cm}|}
    \hline
                   & \centering ΔΕΥΤΕΡΑ & \centering ΤΡΙΤΗ & \centering ΤΕΤΑΡΤΗ & \centering ΠΕΜΠΤΗ & \hspace{.8cm} ΠΑΡΑΣΚΕΥΗ  \\
    \hline
                   &         &       &         &        &           \\
    1η 08:15-09:00 &         &       &         &        &           \\
                   &         &       &         &        &           \\
    \hline
                   &         &       &         &        &           \\
    2η 09:05-09:50 &         &       &         &        &           \\
                   &         &       &         &        &           \\
    \hline
                   &         &       &         &        &           \\
    3η 10:00-10:45 &         &       &         &        &           \\
                   &         &       &         &        &           \\
    \hline
                   &         &       &         &        &           \\
    4η 10:55-11:40 &         &       &         &        &           \\
                   &         &       &         &        &           \\
    \hline
                   &         &       &         &        &           \\
    5η 11:50-12:35 &         &       &         &        &           \\
                   &         &       &         &        &           \\
    \hline
                   &         &       &         &        &           \\
    6η 12:40-13:25 &         &       &         &        &           \\
                   &         &       &         &        &           \\
    \hline
                   &         &       &         &        &           \\
    7η 13:30-14:10 &         &       &         &        &           \\
                   &         &       &         &        &           \\
    \hline
\end{tabular}


%-----------------------------------------------------------
%-----------------------------------------------------------
\newpage
%-----------------------------------------------------------
%-----------------------------------------------------------
\begin{center}
    {\Large \textbf{ΤΥΠΟΛΟΓΙΟ β' τάξης Γυμνασίου}}
\end{center}
%\rule{\textwidth}{0.5pt}

\begin{multicols}{3}
%-----------------------------------------------------------
\myTitle{ΕΞΙΣΩΣΕΙΣ}

\mybullet $a\cdot\beta=0\Leftrightarrow a=0$ ή $\beta=0$ \\
\mybullet $a\cdot\beta\ne0\Leftrightarrow a\ne0$ και $\beta\ne0$

\noindent\textbf{βήματα επίλυσης}\\
%\mybullet Βγάζω παρονομαστές και παρενθέσεις \\
\mybullet Χωρίζω γνωστούς από αγνώστους \\
\mybullet Αναγωγή ομοίων όρων \\
\mybullet Διαιρώ με τον συντελεστή του αγνώστου \\


%-----------------------------------------------------------
\myTitle{Τετραγωνική ρίζα}

\begin{minipage}{.5\linewidth}
\mybullet $\sqrt{a}\ge0$ \\
\mybullet $\sqrt{a^2}=|a|$ \\
\mybullet $\left(\sqrt{a}\right)^2=a$ \\
\mybullet $\sqrt{a}=0\Leftrightarrow a=0$ \\
\end{minipage}
\begin{minipage}{.5\linewidth}
\mybullet $\sqrt{a}\ne0\Leftrightarrow a\ne0$ \\
\mybullet $\sqrt{a\cdot\beta}=\sqrt{a}\cdot \sqrt{\beta}$ \\
\mybullet $\sqrt{\dfrac{a}{\beta}}=\dfrac{\sqrt{a}}{\sqrt{\beta}}$ \\
\end{minipage}


%-----------------------------------------------------------
\myTitle{$y=ax$ με $a\ne0$ (Ευθεία)}

\mybullet τα ποσά $x,y$ είναι ανάλογα. \\
%\mybullet παριστάνει ευθεία. \\
\mybullet περνά από την αρχή των αξόνων. \\
\mybullet τα $a$ ονομάζεται κλίση της ευθείας. \\
\mybullet αν $a>0$ η ευθεία ανεβαίνει. \\
\mybullet αν $a<0$ η ευθεία κατεβαίνει. \\


%-----------------------------------------------------------
\myTitle{$y=ax+\beta$ (Ευθεία)}

\mybullet τα ποσά $x,y$ ΔΕΝ είναι ανάλογα. \\
%\mybullet παριστάνει ευθεία. \\
\mybullet περνά από το $\beta$ του άξονα $y'y$. \\
\mybullet τα $a$ ονομάζεται κλίση της ευθείας. \\
\mybullet αν $a>0$ η ευθεία ανεβαίνει. \\
\mybullet αν $a=0$ η ευθεία είναι οριζόντια. \\
\mybullet αν $a<0$ η ευθεία κατεβαίνει. \\


%-----------------------------------------------------------
\myTitle{$y=\dfrac{a}{x}$ με $a\ne0$ (Υπερβολή)}

\mybullet τα ποσά $x,y$ είναι αντιστρόφως ανάλογα. \\
%\mybullet παριστάνει υπερβολή. \\
\mybullet Το $x$ δεν μπορεί να είναι μηδέν. \\
\mybullet αν $a>0$, περνά από το 1ο, 3ο τεταρτημόριο. \\
\mybullet αν $a<0$, περνά από το 2ο, 4ο τεταρτημόριο. \\


%-----------------------------------------------------------
\myTitle{Τριγωνομετρία}

\mybullet $\hm a=\dfrac{\text{απέναντι κάθετη}}{\text{υποτείνουσα}}$ \\
\mybullet $\syn a=\dfrac{\text{προσκείμενη κάθετη}}{\text{υποτείνουσα}}$ \\
\mybullet $\ef a=\dfrac{\text{απέναντι κάθετη}}{\text{προσκείμενη κάθετη}}$ \\


\newcolumn
%-----------------------------------------------------------
\myTitle{Εμβαδά}

\mybullet Τετράγωνο $E=a^2$, ($a=$ πλευρά). \\
\mybullet Ορθογώνιο $E=a\times\beta$, ($a,\beta=$κάθετες πλευρές). \\
\mybullet Παραλληλόγραμμο $E=$βάση$\times$ύψος. \\
\mybullet Τρίγωνο $E=\dfrac{\text{βάση}\times\text{ύψος}}{2}$ \\
\mybullet Τραπέζιο $E=\dfrac{\text{(βάση μικρή + βάση μεγάλη)}\times\text{ύψος}}{2}$ \\


%-----------------------------------------------------------
\myTitle{Πυθαγόρειο Θεώρημα}

\mybullet $a^2=\beta^2+\gamma^2$, όπου α η υποτείνουσα. \\


%-----------------------------------------------------------
\myTitle{Μέτρηση Κύκλου}

($\pi=3,14$ και $\rho=$ακτίνα) \\
\mybullet Μήκος κύκλου: $L=2\pi\rho$ \\
\mybullet Εμβαδόν κυκλικού δίσκου: $E=\pi\rho^2$ \\

\end{multicols}


%\QuestionNumber Να υπολογίσετε τα πεδία ορισμού των συναρτήσεων: $f(x)=\frac{\sqrt{1-|\ln{x}|}}{|x|+x}$, $\ g(x)=\ln{\big(x+\sqrt{x^2+1}\big)}$, $h(x)=\ln{\big(x^2-4|x|-5\big)}$,  $\ s(x)=\sqrt{\ln{(e^x+2)}-x}+\ln{|x|}$, $\ t(x)=\frac{1}{2\syn x-1}$, $\ u(x)=\sqrt{2\hm x-1}$.

\end{document}
